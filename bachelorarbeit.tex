% Vorlage für eine Bachelorarbeit - 2012-2013 Timo Bingmann

% Dies ist nur eine Vorlage. Strikte Vorgaben wie die Bachelorarbeit auszusehen
% hat gibt es nicht. Darum können auch alle Teile angepasst werden.

%\documentclass[12pt,a4paper,twoside]{scrartcl}
\documentclass[a4paper,12pt,bibtotoc,titlepage, liststotoc,BCOR7mm,headsepline,pointlessnumbers]{scrbook}

% Diese (und weitere) Eingabedateien sind in UTF-8
\usepackage[utf8]{inputenc}

\usepackage[T1]{fontenc}
\usepackage{times}
\typearea[current]{current}
\usepackage{fancyhdr}
\newenvironment{code}{\begingroup\footnotesize \verbatim}{\endverbatim\endgroup}

\addtokomafont{caption}{\small} % Aktive Auszeichnung von Legenden
\setkomafont{captionlabel}{\sffamily\bfseries}
\bibliographystyle{plain}

% Sprache des Dokuments (für Silbentrennung und mehr)
\usepackage[english,german]{babel}

% Einrückung und Abstand zwischen Paragraphen
\setlength\parskip{\smallskipamount}
\setlength\parindent{0pt}

% Einige Standard-Mathematik Pakete
\usepackage{latexsym,amsmath,amssymb,mathtools,textcomp}

% Unterstützung für Sätze und Definitionen
\usepackage{amsthm}

\newtheorem{Satz}{Satz}[section]
\newtheorem{Definition}[Satz]{Definition}
\newtheorem{Lemma}[Satz]{Lemma}

\numberwithin{equation}{section}

% Unterstützung zum Einbinden von Graphiken
\usepackage{graphicx}

% Pakete die tabular und array verbessern
\usepackage{array,multirow}

% Kleiner enumerate und itemize Umgebungen
\usepackage{enumitem}

\setlist[enumerate]{topsep=0pt}
\setlist[itemize]{topsep=0pt}
\setlist[description]{font=\normalfont,topsep=0pt}

\setlist[enumerate,1]{label=(\roman*)}

% TikZ für Graphiken in LaTeX
\usepackage{tikz}
\usetikzlibrary{calc}

% Hyperref für Hyperlink und Sprungtexte
\usepackage{xcolor}

\usepackage[plainpages=false,pdfpagelabels=false,citecolor=Black, linkcolor=Black]{hyperref} %Verweise werden Links im PDF

% Paket zum Setzen von Algorithmen in Pseudocode mit kleinen Stilanpassungen
\usepackage[ruled,vlined,linesnumbered,norelsize]{algorithm2e}
\DontPrintSemicolon
\def\NlSty#1{\textnormal{\fontsize{8}{10}\selectfont{}#1}}
\SetKwSty{texttt}
\SetCommentSty{emph}
\def\listalgorithmcfname{Algorithmenverzeichnis}
\def\algorithmautorefname{Algorithmus}

\begin{document}

%%%%%%%%%%%%%%%%%%%%%%%%%%%%%%%%%%%%%%%%%%%%%%%%%%%%%%%%%%%%%%%%%%%%%%

\pagestyle{empty} % keine Seitenzahlen
\renewcommand{\thepage}{\roman{page}}

% Titelblatt der Arbeit
\begin{titlepage}

  \begin{center}\large
  \begin{flushleft}
    \quad\includegraphics[height=17mm]{kit_logo_de.pdf} \hfill
  \end{flushleft}
    %\includegraphics[height=20mm]{grouplogo-algo-blue.pdf}\quad\null

    \vfill
    \vfill
    \vfill
    \vfill

    Bachelor thesis
    \vspace*{2cm}

    {\bf\huge Title of Thesis  \par}
    % Siehe auch oben die Felder pdftitle={}
    % mit \par am Ende stimmt der Zeilenabstand

    \vfill

    Name of author

    \vspace*{15mm}

    Date: \today 

    \vspace*{40mm}
    \begin{tabular}{rl}
      Supervisors: & Prof. Dr. Peter Sanders \\
      & Dipl. Inform. Zweiter Betreuer \\
    \end{tabular}
    
    \vspace*{10mm}

    %Institut für Theoretische Informatik, Algorithmik \\
    %Fakultät für Informatik \\
    %Karlsruher Institut für Technologie

    \vspace*{10mm}
    % English:
     Institute of Theoretical Informatics, Algorithmics \\
     Department of Informatics \\
     Karlsruhe Institute of Technology

    \vspace*{12mm}
    \vfill
  \end{center}

\end{titlepage}

%%%%%%%%%%%%%%%%%%%%%%%%%%%%%%%%%%%%%%%%%%%%%%%%%%%%%%%%%%%%%%%%%%%%%%
%%%%%%%%%%%%%%%%%%%%%%%%%%%%%%%%%%%%%%%%%%%%%%%%%%%%%%%%%%%%%%%%%%%%%%

%\vspace*{0pt}\vfill
\ 
\newpage
\clearpage

\section*{Abstract}
In this thesis we augment the existing Hypergraph Partitioner KaHyPar with an evolutionary framework with the goal to improve the 
solution quality.
\addcontentsline{toc}{chapter}{Abstract}
\selectlanguage{english}
% German and English Abstract
\vfill\vfill\vfill
\ 
\newpage
\clearpage
\ 
\newpage
\clearpage

%%%%%%%%%%%%%%%%%%%%%%%%%%%%%%%%%%%%%%%%%%%%%%%%%%%%%%%%%%%%%%%%%%%%%%
\section*{Acknowledgments}

I'd like to thank Timo for the supply of Club-Mate
\vfill\vfill\vfill
Hiermit versichere ich, dass ich diese Arbeit selbständig verfasst und keine anderen, als die angegebenen Quellen und Hilfsmittel benutzt, die wörtlich oder inhaltlich übernommenen Stellen als solche kenntlich gemacht und die Satzung des Karlsruher Instituts für Technologie zur Sicherung guter wissenschaftlicher Praxis in der jeweils gültigen Fassung beachtet habe.

\bigskip
\vspace*{1cm}
\noindent
Ort, den Datum

\clearpage

%%%%%%%%%%%%%%%%%%%%%%%%%%%%%%%%%%%%%%%%%%%%%%%%%%%%%%%%%%%%%%%%%%%%%%
\tableofcontents
\clearpage
%%%%%%%%%%%%%%%%%%%%%%%%%%%%%%%%%%%%%%%%%%%%%%%%%%%%%%%%%%%%%%%%%%%%%%
\clearpage
%%%%%%%%%%%%%%%%%%%%%%%%%%%%%%%%%%%%%%%%%%%%%%%%%%%%%%%%%%%%%%%%%%%%%%
\mainmatter
\pagestyle{plain}
\chapter{Introduction}
\pagestyle{headings}
\section{Motivation}
Hypergraph Partitioning is a highly complex field of study. The Motivation behind this work is to add an evolutionary
framework to the existing Hypergraph partitioner KaHyPar in order to improve the cuts of the Hypergraph.
\section{Contribution}
\section{Structure of Thesis}
%%%%%%%%%%%%%%%%%%%%%%%%%%%%%%%%%%%%%%%%%%%%%%%%%%%%%%%%%%%%%%%%%%%%%%
\chapter{Fundamentals}
\section{General Definitions}
A Hypergraph $H = (V, E, c, w)$ is defined as a set of verticies $V$ a set of hyperedges $E$ where each edge may contain an arbitrarily large subset of $V$.
$c: V \rightarrow  \mathbb R_{\ge 0}$ is a function applying a weight to each Vertex and $w: E \rightarrow  \mathbb R_{\ge 0}$ applying a weight to each hyperedge. 
Two verticies $u, v$ are adjacent if $\exists e \in E | u, v \in e$ and a vertex $u$ is incident to a hyperedge $e$ if $ u \in e$. The size $|e|$ of an Hyperedge $e$ is the number of verticies contained in $e$. A k-way partition of a Hypergraph $H$ is a partition of $V$ into k disjoint blocks $V_1, .. V_k$. $part: V \rightarrow [0, k-1]$ is a function referencing the corresponding block of a k-way partition to a vertex $u$. 
A k-way partition is balanced when the weight of each block $V_i | 1 \le i \le k, \sum_{v_i \in V_i} c(v_i) \le (1 + \epsilon) \lceil \frac{\sum_{v \in V} c(v)}{k} \rceil $  for a balance constraint $\epsilon$.
A valid solution is a balanced k-way partition. An invalid solution is a partition where either the balance criterion is not met, or $H$ has been partitioned for a different value for k.
A Hyperedge $e$ is a cut edge $cut(e)$ if $\exists u,v \in e | part(u) \neq part(v)$. The connectivity of a Hyperedge $e$ is $\lambda(e)$ = $\sum_{i=0}^{k-1} \delta(e,i) | \delta(e, i) = 
     \begin{cases}
       \text{1} &\exists v \in e\text{ }part(v)=i\\
       \text{0} &\text{else}\\
     \end{cases}$ 

The set cut edges in H is defined as
$cut(E) := \{e \in E | cut(e) \}$.
The multiset connectivity edges in H is defined as
$conn(E) := \{a(e) \in E | cut(e) \} |a(e) := \lambda(e)$
The cut metric $cut(H) := \sum_{e \in E} 
\begin{cases}
       \text{w(e)} & \text{e is cut edge}\\
       \text{0} &\text{else}\\
     \end{cases}$  and gives the value of cuts.
The connectivity metric $(\lambda -1)(H) := \begin{cases}
       \text{$\lambda(e)*w(e)$} & \text{e is cut edge}\\
       \text{0} &\text{else}\\
     \end{cases}$
Both metrics can be used to measure the quality of a solution. Throughout this thesis the solution quality is referenced. The metrics are interchangeable in this regard.
An Individual $I$ is a valid solution for the k-way partition problem of $H$.
An individual eligible for further operations is considered $alive$.
The difference of two individuals $I_1, I_2$ is $diff(I_1, I_2) := cut(I_1) \ominus cut(I_2)$
The connectivity difference of two individuals $I_1, I_2$ is $strongdiff(I_1, I_2) := conn(I_1) \ominus conn(I_2)$
A population $P$ is a collection of Individuals.

%Hypergraph := (H;V;E;C)
%Node contraction := 
%Node uncontraction := 
%Net :=
%cut :=
%onnectivity :=
%quality :=
%p%opulation :=
%individual :=
%solution := 
%mbalance:= 
\chapter{Related Work}
The Hypergraph Partitioner KaHyPar uses a multilevel approach for partition. The original Hypergraph $H$ is coarsened by contracting two nodes $u,v$ using 
a selection strategy until a certain limit is passed. %TODO explain node selection for coarsening 
On the coarsened Hypergraph a simpler partitioning algorithm is chosen to generate an initial partitioning. Afterwards to contraction operations will be reverted and 
during each step of the uncoarsening phase local search algorithms are used to improve the current connectivity of $H$. The local search is based on the principle of
Fiduccia-Mattheysis algorithm where operations of decreasing quality are considered, but only executed if the sum of operations results in a net gain.
\newline
The memetic Graph Partitioner KaHiP uses evolutionary actions to increase solution quality and is the main inspiration for the operations implemented in KaHyParE.
%KaHyPar
%KaHiP 
%edge-frequency
%%%%%%%%%%%%%%%%%%%%%%%%%%%%%%%%%%%%%%%%%%%%%%%%%%%%%%%%%%%%%%%%%%%%%%
%TODO only one individual is generated per iteration
%TODO explain evolutionary algorithms, or at least show that this is one
\chapter{KaHyParE}
\section{Overview}
The objective of the algorithm is to optimize solution quality of a k-way partition of $H$ within bounded time. During this time
the algorithm will operate on already existing solutions in order to generate improved solutions. 
\section{Population}
The algorithm will produce multiple individuals, which are inserted and removed from the population. At any given time only a finite amount of
individuals are $alive$ which is the maximum population size. Furhter individuals have to compete for a place in the population. 
The population size is an important parameter, as a small value limits the solution scope and a high value limits convergence.
We use KaHyPar to fill the initial population.
in order to select a proper population size for the runtime, we dynamically allocate 15\% of the runtime to create and fill the initial population.
\section{Diversity}
By measuring the difference between two individuals we can gain some knowledge over their internal structure and similarities.
using the edges rather than the verticies for difference ensures that only elements relevant for the solution quality are considered
for difference and perturbations of the partition blocks are not influencing the difference. As such we can consider the diversity of
the population as the difference of the current individuals. Therefore a low diversity is a population where the individuals are 
in close proximity considering the solution space or convergent. A high diversity means that the individuals are spread throughout a 
larger portion of the solution space.  
\section{Selection Strategies}
For an evolutionary algorithm we attempt to generate new, improved solutions by using existing solutions. A logical conclusion is that good individuals probably will generate solutions
close to the original individual and as such good. We select our individuals using tournament selection %TODO explain tournament selection 
\section{Combine operations}
\subsection{Basic Combine}
The basic combine uses two parent partitions $P_1$ $P_2$ in order to create a child individual $C$. This is achieved by only allowing contractions of nodes $u, v$ when these nodes are not placed in different partitions in either parent. Afterwards we do not perform an initial partitioning, instead we consider the coarsened Hypergraph and see which of the parents gives the better objective on said graph. 
Since the contraction condition is rather strong and we do not need to do an initial partitioning, we can remove the coarsening limit $s$ used by KaHyPar
to allow for more contractions. The uncoarsening and application of local search algorithms which guarantee to atleast maintain solution quality in combination with using the better partition of the two parents ensures that the child solution is at least as good as the best parent solution. It is imporant to note that the combine operation is more powerful the more diverse the parent individuals are, since similar parents are essentially just a vcycle. This will be important in the replacement strategy which will insert the child into the population. 
\subsection{Cross Combine}

\subsection{Edge Frequency Multicombine}
\subsection{Basic Combine + Edge Frequency Information}
\section{Mutation operations}
\subsection{VCycle}
\subsection{VCycle + New Initial Partitioning}
\subsection{Stable Nets}
\section{Replacement Strategies}
%.....
%%%%%%%%%%%%%%%%%%%%%%%%%%%%%%%%%%%%%%%%%%%%%%%%%%%%%%%%%%%%%%%%%%%%%%
\chapter{Experimental Evaluation}
\section{Implementation}
\section{Experimental Setup}
\subsection{Environment}
\subsection{Tuning Parameters}
\subsection{Instances}
\section{Your Experiment Headline}
%%%%%%%%%%%%%%%%%%%%%%%%%%%%%%%%%%%%%%%%%%%%%%%%%%%%%%%%%%%%%%%%%%%%%%
\chapter{Discussion}
\section{Conclusion}
\section{Future Work}

\clearpage
\begin{appendix}
\chapter{Implementation Details}
\section{Software}
% Data Structures etc..
\section{Hardware}
\end{appendix}
%%%%%%%%%%%%%%%%%%%%%%%%%%%%%%%%%%%%%%%%%%%%%%%%%%%%%%%%%%%%%%%%%%%%%%
\bibliographystyle{gerplain}
\bibliography{literatur}
\end{document}
