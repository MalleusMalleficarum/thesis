% Vorlage für eine Bachelorarbeit - 2012-2013 Timo Bingmann

% Dies ist nur eine Vorlage. Strikte Vorgaben wie die Bachelorarbeit auszusehen
% hat gibt es nicht. Darum können auch alle Teile angepasst werden.

%\documentclass[12pt,a4paper,twoside]{scrartcl}
\documentclass[a4paper,12pt,bibtotoc,titlepage, liststotoc,BCOR7mm,headsepline,pointlessnumbers]{scrbook}

% Diese (und weitere) Eingabedateien sind in UTF-8
\usepackage[utf8]{inputenc}

\usepackage[T1]{fontenc}
\usepackage{times}
\typearea[current]{current}
\usepackage{fancyhdr}
\newenvironment{code}{\begingroup\footnotesize \verbatim}{\endverbatim\endgroup}

\addtokomafont{caption}{\small} % Aktive Auszeichnung von Legenden
\setkomafont{captionlabel}{\sffamily\bfseries}
\bibliographystyle{plain}

% Sprache des Dokuments (für Silbentrennung und mehr)
\usepackage[english,german]{babel}

% Einrückung und Abstand zwischen Paragraphen
\setlength\parskip{\smallskipamount}
\setlength\parindent{0pt}

% Einige Standard-Mathematik Pakete
\usepackage{latexsym,amsmath,amssymb,mathtools,textcomp}

% Unterstützung für Sätze und Definitionen
\usepackage{amsthm}

\newtheorem{Satz}{Satz}[section]
\newtheorem{Definition}[Satz]{Definition}
\newtheorem{Lemma}[Satz]{Lemma}

\numberwithin{equation}{section}

% Unterstützung zum Einbinden von Graphiken
\usepackage{graphicx}

% Pakete die tabular und array verbessern
\usepackage{array,multirow}

% Kleiner enumerate und itemize Umgebungen
\usepackage{enumitem}

\setlist[enumerate]{topsep=0pt}
\setlist[itemize]{topsep=0pt}
\setlist[description]{font=\normalfont,topsep=0pt}

\setlist[enumerate,1]{label=(\roman*)}

% TikZ für Graphiken in LaTeX
\usepackage{tikz}
\usetikzlibrary{calc}

% Hyperref für Hyperlink und Sprungtexte
\usepackage{xcolor}

\usepackage[plainpages=false,pdfpagelabels=false,citecolor=Black, linkcolor=Black]{hyperref} %Verweise werden Links im PDF

% Paket zum Setzen von Algorithmen in Pseudocode mit kleinen Stilanpassungen
\usepackage[ruled,vlined,linesnumbered,norelsize]{algorithm2e}
\DontPrintSemicolon
\def\NlSty#1{\textnormal{\fontsize{8}{10}\selectfont{}#1}}
\SetKwSty{texttt}
\SetCommentSty{emph}
\def\listalgorithmcfname{Algorithmenverzeichnis}
\def\algorithmautorefname{Algorithmus}

\begin{document}

%%%%%%%%%%%%%%%%%%%%%%%%%%%%%%%%%%%%%%%%%%%%%%%%%%%%%%%%%%%%%%%%%%%%%%

\pagestyle{empty} % keine Seitenzahlen
\renewcommand{\thepage}{\roman{page}}

% Titelblatt der Arbeit
\begin{titlepage}

  \begin{center}\large
  \begin{flushleft}
    \quad\includegraphics[height=17mm]{kit_logo_de.pdf} \hfill
  \end{flushleft}
    %\includegraphics[height=20mm]{grouplogo-algo-blue.pdf}\quad\null

    \vfill
    \vfill
    \vfill
    \vfill

    Bachelor thesis
    \vspace*{2cm}

    {\bf\huge Title of Thesis  \par}
    % Siehe auch oben die Felder pdftitle={}
    % mit \par am Ende stimmt der Zeilenabstand

    \vfill

    Name of author

    \vspace*{15mm}

    Date: \today 

    \vspace*{40mm}
    \begin{tabular}{rl}
      Supervisors: & Prof. Dr. Peter Sanders \\
      & Dipl. Inform. Zweiter Betreuer \\
    \end{tabular}
    
    \vspace*{10mm}

    %Institut für Theoretische Informatik, Algorithmik \\
    %Fakultät für Informatik \\
    %Karlsruher Institut für Technologie

    \vspace*{10mm}
    % English:
     Institute of Theoretical Informatics, Algorithmics \\
     Department of Informatics \\
     Karlsruhe Institute of Technology

    \vspace*{12mm}
    \vfill
  \end{center}

\end{titlepage}

%%%%%%%%%%%%%%%%%%%%%%%%%%%%%%%%%%%%%%%%%%%%%%%%%%%%%%%%%%%%%%%%%%%%%%
%%%%%%%%%%%%%%%%%%%%%%%%%%%%%%%%%%%%%%%%%%%%%%%%%%%%%%%%%%%%%%%%%%%%%%

%\vspace*{0pt}\vfill
\ 
\newpage
\clearpage

\section*{Abstract}
In this thesis we augment the existing Hypergraph Partitioner KaHyPar with an evolutionary framework with the goal to improve the 
solution quality.
\addcontentsline{toc}{chapter}{Abstract}
\selectlanguage{english}
% German and English Abstract
\vfill\vfill\vfill
\ 
\newpage
\clearpage
\ 
\newpage
\clearpage

%%%%%%%%%%%%%%%%%%%%%%%%%%%%%%%%%%%%%%%%%%%%%%%%%%%%%%%%%%%%%%%%%%%%%%
\section*{Acknowledgments}

I'd like to thank Timo for the supply of Club-Mate
\vfill\vfill\vfill
Hiermit versichere ich, dass ich diese Arbeit selbständig verfasst und keine anderen, als die angegebenen Quellen und Hilfsmittel benutzt, die wörtlich oder inhaltlich übernommenen Stellen als solche kenntlich gemacht und die Satzung des Karlsruher Instituts für Technologie zur Sicherung guter wissenschaftlicher Praxis in der jeweils gültigen Fassung beachtet habe.

\bigskip
\vspace*{1cm}
\noindent
Ort, den Datum

\clearpage

%%%%%%%%%%%%%%%%%%%%%%%%%%%%%%%%%%%%%%%%%%%%%%%%%%%%%%%%%%%%%%%%%%%%%%
\tableofcontents
\clearpage
%%%%%%%%%%%%%%%%%%%%%%%%%%%%%%%%%%%%%%%%%%%%%%%%%%%%%%%%%%%%%%%%%%%%%%
\clearpage
%%%%%%%%%%%%%%%%%%%%%%%%%%%%%%%%%%%%%%%%%%%%%%%%%%%%%%%%%%%%%%%%%%%%%%
\mainmatter
\pagestyle{plain}
\chapter{Introduction}
\pagestyle{headings}
\section{Motivation}
Hypergraph Partitioning is a highly complex field of study. The Motivation behind this work is to add an evolutionary
framework to the existing Hypergraph partitioner KaHyPar in order to improve the cuts of the Hypergraph.
\section{Contribution}
\section{Structure of Thesis}
%%%%%%%%%%%%%%%%%%%%%%%%%%%%%%%%%%%%%%%%%%%%%%%%%%%%%%%%%%%%%%%%%%%%%%
\chapter{Fundamentals}
\section{General Definitions}
Hypergraph := (H;V;E;C)
Node contraction := 
Node uncontraction := 
Net :=
cut :=
connectivity :=
quality :=
population :=
individual :=
solution := 
imbalance:= 
\chapter{Related Work}
The Hypergraph Partitioner KaHyPar uses a multilevel approach for partition. The original Hypergraph $H$ is coarsened by contracting two nodes $u,v$ using 
a selection strategy until a certain limit is passed. %TODO explain node selection for coarsening 
On the coarsened Hypergraph a simpler partitioning algorithm is chosen to generate an initial partitioning. Afterwards to contraction operations will be reverted and 
during each step of the uncoarsening phase local search algorithms are used to improve the current connectivity of $H$. The local search is based on the principle of
Fiduccia-Mattheysis algorithm where operations of decreasing quality are considered, but only executed if the sum of operations results in a net gain.
\newline
The memetic Graph Partitioner KaHiP uses evolutionary actions to increase solution quality and is the main inspiration for the operations implemented in KaHyParE.
%KaHyPar
%KaHiP 
%edge-frequency
%%%%%%%%%%%%%%%%%%%%%%%%%%%%%%%%%%%%%%%%%%%%%%%%%%%%%%%%%%%%%%%%%%%%%%
\chapter{KaHyParE}
\section{Overview}
The main objective of the algorithm is to optimize solution quality within bounded time. The algorithm will produce multiple solutions 
\section{Population}
We manage a Population $P$ which is the resource for the partitions.
An individual $I$ is a member of the population, if it is a valid solution
for the given Hypergraph $H$. 
\section{Diversity}
We define diversity as 
\section{Selection Strategies}

\section{Combine operations}
\subsection{Basic Combine}
The basic combine uses two parent partitions $P_1$ $P_2$ in order to create a child individual $C$. This is achieved by only allowing contractions of nodes $u, v$ when these nodes are not placed in different partitions in either parent. Afterwards we do not perform an initial partitioning, instead we consider the coarsened Hypergraph and see which of the parents gives the better objective on said graph. 
Since the contraction condition is rather strong and we do not need to do an initial partitioning, we can remove the coarsening limit $s$ 
to allow for more contractions. The uncoarsening and application of local search algorithms which guarantee to atleast maintain solution quality in combination with using the better partition of the two parents ensures that the child solution is at least as good as the best parent solution. It is imporant to note that the combine operation is more powerful the more diverse the parent individuals are. This will be important in the replacement strategy which will insert the child into the population. 
\subsection{Cross Combine}
\subsection{Edge Frequency Multicombine}
\subsection{Basic Combine + Edge Frequency Information}
\section{Mutation operations}
\subsection{VCycle}
\subsection{VCycle + New Initial Partitioning}
\subsection{Stable Nets}
\section{Replacement Strategies}
%.....
%%%%%%%%%%%%%%%%%%%%%%%%%%%%%%%%%%%%%%%%%%%%%%%%%%%%%%%%%%%%%%%%%%%%%%
\chapter{Experimental Evaluation}
\section{Implementation}
\section{Experimental Setup}
\subsection{Environment}
\subsection{Tuning Parameters}
\subsection{Instances}
\section{Your Experiment Headline}
%%%%%%%%%%%%%%%%%%%%%%%%%%%%%%%%%%%%%%%%%%%%%%%%%%%%%%%%%%%%%%%%%%%%%%
\chapter{Discussion}
\section{Conclusion}
\section{Future Work}

\clearpage
\begin{appendix}
\chapter{Implementation Details}
\section{Software}
% Data Structures etc..
\section{Hardware}
\end{appendix}
%%%%%%%%%%%%%%%%%%%%%%%%%%%%%%%%%%%%%%%%%%%%%%%%%%%%%%%%%%%%%%%%%%%%%%
\bibliographystyle{gerplain}
\bibliography{literatur}
\end{document}
