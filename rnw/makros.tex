% anderweitig wiederverwendbare Makros fuer Diss
\usepackage{amsfonts}

% allgemeine mathematische Notation
\newcommand{\ceil}[1]{\left\lceil #1\right\rceil}
\newcommand{\floor}[1]{\left\lfloor #1\right\rfloor}
\newcommand{\abs}[1]{\left| #1\right|}
\newcommand{\seq}[1]{\langle #1\rangle}
\newcommand{\norm}[1]{\left\|#1\right\|}
\newcommand{\enorm}[1]{\norm{#1}_{2}}
\newcommand{\sumnorm}[1]{\norm{#1}_{1}}
\newcommand{\maxnorm}[1]{\norm{#1}_{\infty}}
\newcommand{\xor}{\oplus}
\newcommand{\set}[1]{\left\{ #1\right\}}
\newcommand{\gilt}{:}
\newcommand{\sodass}{\,:\,}
\newcommand{\setGilt}[2]{\left\{ #1\sodass #2\right\}}
\newcommand{\Def}{:=}
\newcommand{\zvektor}[2]{\left(#1,#2\right)}
\newcommand{\vektor}[2]{\left(\begin{smallmatrix}#1\\#2\end{smallmatrix}\right)}
\newcommand{\condition}[1]{\left[#1\right]}
\newcommand{\binomial}[2]{\binom{#1}{#2}}
\newcommand{\even}{\mathrm{even}}
\newcommand{\odd}{\mathrm{odd}}
\newcommand{\mymod}{\,\bmod\,}
\newcommand{\divides}{|}

\newcommand{\blank}{\Box}
\newcommand{\cross}[2]{\langle #1,#2 \rangle}
\newcommand{\defeq}{\mathrel{:=}}
\newcommand{\hash}{\sym{@}}
\newcommand{\hD}[1][D]{^{(#1)}}
\newcommand{\hed}[1][D]{^{1/#1}}
\newcommand{\Lpal}{L_{\mathrm{pal}}}
\newcommand{\Lparity}{L_{\mathrm{parity}}}
\newcommand{\Lvv}{L_{\mathrm{vv}}}
\newcommand{\pD}[1][D]{^{[#1]}}
\newcommand{\pos}[1]{\mathbf{#1}}
\newcommand{\qs}{\mathord{\Box}}
\newcommand{\mSet}[2]{\left\{#1 \mid #2\right\}}
\newcommand{\s}{\mathord{-}}
\newcommand{\CAce}{\hbox{\mdseries\scshape CAce}}
\newcommand{\sCAce}{\s\CAce}
\newcommand{\CAcs}{\hbox{\mdseries\scshape CAcs}}
\newcommand{\sCAcs}{\s\CAcs}
\newcommand{\CAww}{\hbox{\mdseries\scshape CAww}}
\newcommand{\sCAww}{\s\CAww}
\newcommand{\sC}{\s\hbox{\scshape Chng}}
\newcommand{\sD}{\s\hbox{\scshape Diam}}
\newcommand{\sT}{\s\hbox{\scshape Time}}
\newcommand{\CA}{\hbox{\mdseries\scshape CA}}
\newcommand{\sCA}{\s\CA}
\newcommand{\ST}[1]{\langle #1\rangle}
\newcommand{\sym}[1]{\mathord{\hbox{\texttt{\upshape #1}}}}

% Typen
\newcommand{\nat}{\mathbb{N}}
\newcommand{\natnull}{\mathbb{N}_{0}}
%\newcommand{\natless}[1]{\mathbb{N}_{<#1}}
\newcommand{\natless}[1]{\mathbb{N}_{#1}}
\newcommand{\nplus}{\mathbb{N}_+}
\newcommand{\real}{\mathbb{R}}
\newcommand{\rplus}{\mathbb{R}_+}
\newcommand{\rnneg}{\mathbb{R}_*}
\newcommand{\integer}{\mathbb{Z}}
% \newcommand{\intint}[2]{\set{#1,\ldots, #2}}
\newcommand{\intint}[2]{{#1}..{#2}}
\newcommand{\realrange}[2]{\left[#1, #2\right]}
\newcommand{\realrangeo}[2]{\left(#1, #2\right)}
\newcommand{\realrangelo}[2]{\left(#1, #2\right]}
\newcommand{\realrangero}[2]{\left[#1, #2\right)}
\newcommand{\unitrange}[2]{\realrange{0}{1}}
\newcommand{\bool}{\set{0,1}}
%\newcommand{\boolean}{\mathbb{B}}
%\newcommand{\mapping}[2]{#1\rightarrow #2}
\newcommand{\mapping}[2]{{#2}^{#1}}
\newcommand{\powerset}[1]{{\cal P}\left(#1\right)}
\newcommand{\NP}{\mathbf{NP}}
\newcommand{\Bild}{\mathbf{Bild}\:}

% Typannotation
\newcommand{\withtype}[1]{\in#1}

% Wahrscheinlichkeitsrechnung
\newcommand{\prob}[1]{{\mathbf{P}}\left[#1\right]}
\newcommand{\condprob}[2]{{\mathbf{P}}\left[#1\;|\;#2\right]}
\newcommand{\condexpect}[2]{{\mathbf{E}}\left[#1\;|\;#2\right]}
\newcommand{\expect}{{\mathbf{E}}}
\newcommand{\var}{{\mathbf{Var}}}
\newcommand{\quant}[2]{\tilde{#1}_{#2}}

% asymptotische Notation
\newcommand{\whpO}[1]{\tilde{\mathrm{O}}\left( #1\right)}
\newcommand{\Oschlange}{$\tilde{\mathrm{O}}$}
\newcommand{\Ohh}[1]{\mathcal{O}\!\left( #1\right)}
\newcommand{\Oh}[1]{\mathcal{O}\!\left( #1\right)}
\newcommand{\oh}[1]{\mathrm{o}\!\left( #1\right)}
\newcommand{\Th}[1]{\Theta\!\left( #1\right)}
\newcommand{\Thsmall}[1]{\Theta( #1)}
\newcommand{\Om}[1]{\Omega\!\left( #1\right)}
\newcommand{\om}[1]{\omega\!\left( #1\right)}
\newcommand{\Oleq}{\preceq}

% local reference
\newcommand{\lref}[1]{\ref{\labelprefix:#1}}
\newcommand{\llabel}[1]{\label{\labelprefix:#1}}
\newcommand{\labelprefix}{} % later redefined using renewcommand

% Diskussion
\newcommand{\discussionsize}{\small}
\newenvironment{discussion}{\par\discussionsize}{\par}

% open issues
%\marginparwidth5cm
\marginparpush2mm
%\newcommand{\frage}[1]{\makebox[0cm]{$\bigotimes$}\marginpar{\tiny #1}}
%\newcommand{\frage}[1]{{\sf[ #1]}\marginpar{?}}
\newcommand{\frage}[1]{}

\newcommand{\mysubsubsection}[1]{\vspace{2mm}\noindent{\bf #1 }}

% punkt am ende von display math
\newcommand{\punkt}{\enspace .}

% Pseudocode Unterst\"utzung
\newenvironment{code}{\noindent%\sf%
\begin{tabbing}%
\hspace{2em}\=\hspace{2em}\=\hspace{2em}\=\hspace{2em}\=\hspace{2em}\=%
\hspace{2em}\=\hspace{2em}\=\hspace{2em}\=\hspace{2em}\=\hspace{2em}\=%
\kill}{\end{tabbing}}

% 1=pos, 2=llable, 3=caption
\newcommand{\labelcommand}{}
\newcommand{\captiontext}{}
\newsavebox{\codeparam}
\newcounter{lineNumber}
\newenvironment{disscodepos}[3]{%
\renewcommand{\labelcommand}{#2}%
\renewcommand{\captiontext}{#3}%
\sbox{\codeparam}{\parbox{\textwidth}{#3}}%
\begin{figure}[#1]\begin{center}\begin{code}\setcounter{lineNumber}{1}}{%
\end{code}\end{center}\caption{\llabel{\labelcommand}\captiontext}\end{figure}}

\newenvironment{disscode}[2]{\begin{disscodepos}{htb}{#1}{#2}}%
{\end{disscodepos}}

% code in text 
%\newcommand{\codel}[1]{{\sf #1}}
%\newcommand{\codem}[1]{\mathsf{#1}}
\newcommand{\codel}[1]{\mbox{\rm "`#1"'}}
\newcommand{\codem}[1]{\mathrm{#1}}

\newcommand{\id}{\tt}
\newcommand{\Function} {{\bf Function\ }}
\newcommand{\Procedure}{{\bf Procedure\ }}
\newcommand{\Process}{{\bf process\ }}
\newcommand{\While}    {{\bf while\ }}
\newcommand{\Repeat}   {{\bf repeat\ }}
\newcommand{\Until}    {{\bf until\ }}
\newcommand{\Loop}     {{\bf loop\ }}
\newcommand{\Exit}     {{\bf exit\ }}
\newcommand{\Goto}     {{\bf goto\ }}
\newcommand{\Do}       {{\bf do\ }}
\newcommand{\Od}       {{\bf od\ }}
\newcommand{\Dopar}       {{\bf dopar\ }}
\newcommand{\For}      {{\bf for\ }}
\newcommand{\Step}      {{\bf step\ }}
\newcommand{\Foreach}      {{\bf foreach\ }}
\newcommand{\Rof}      {{\bf rof\ }}
\newcommand{\Forall}      {{\bf forall\ }}
\newcommand{\To}       {{\bf to\ }}
\newcommand{\If}       {{\bf if\ }}
\newcommand{\Is}       {:=}
\newcommand{\Endif}    {{\bf endif\ }}
\newcommand{\Fi}       {{\bf fi\ }}
\newcommand{\Then}     {{\bf then\ }}
\newcommand{\Else}     {{\bf else\ }}
\newcommand{\Elsif}    {{\bf elsif\ }}
\newcommand{\Return}   {{\bf return\ }}
\newcommand{\Set}      {{\bf set\ }}
\newcommand{\Boolean}  {{\bf boolean\ }}
\newcommand{\Integer}  {$\integer$}
\newcommand{\True}     {{\bf true\ }}
\newcommand{\False}    {{\bf false\ }}
\newcommand{\Bitand}   {{\bf bitand\ }}
\newcommand{\Var}      {{\bf var\ }}
\newcommand{\Xor}       {{\bf\ xor\ }}
\newcommand{\Not}       {{\bf\ not\ }}
\newcommand{\Or}       {{\bf\ or\ }}
\newcommand{\Div}       {{\bf\ div\ }}
\newcommand{\Mod}       {{\bf\ mod\ }}
\newcommand{\End}       {{\bf end\ }}
\newcommand{\Endfor}       {{\bf endfor\ }}
%% \newcommand{\Rem}[1]   {{\bf (*~}{\rm#1}{\bf ~*)}}
% rechtsbuendiger Kommentar
%\newcommand{\RRem}[1]   {\`{$\mathbf{(*}$~ }{\rm#1}{~$\mathbf{*)}$}}
\newcommand{\RRem}[1]   {\`{\bf --\hspace{0.5mm}--~}{\rm#1}}
\newcommand{\RRemNL}[1]   {\`{\bf (*~ }{\rm#1}{\bf ~*)}%
{\tiny\arabic{lineNumber}}\stepcounter{lineNumber}}

\newcommand{\At}[1]{\left\langle#1\right\rangle}
\newcommand{\NL}{\`{\tiny\arabic{lineNumber}}\stepcounter{lineNumber}}

% Parallelverarbeitungspseudocode
\newcommand{\iProc}{i_\mathrm{PE}}

% Parameter 1=pos, 2=xsize, 3=filename, 4=llabel, 4=caption
\newcommand{\dissepslong}[5]{\begin{figure}[#1]\begin{center}%
\epsfxsize#2\leavevmode\epsfbox{#3.eps}%
\end{center}\caption{\llabel{#4}#5}\end{figure}}

\newcommand{\dissepspos}[4]{\dissepslong{#1}{#2}{\labelprefix/#3}{#3}{#4}}
\newcommand{\disseps}[3]{\dissepspos{htb}{#1}{#2}{#3}}

% Beweise
\newdimen\endofsize\endofsize=0.5em
\def\endofbeweis{~\quad\hglue\hsize minus\hsize
                 \hbox{\vrule height \endofsize width
\endofsize}\par}
% gibt es in amsmath schon
\newenvironment{myproof}{\begin{proof}}{\endofbeweis\end{proof}}
% \newcommand{\platsch}{\hglue\hsize minus\hsize}

